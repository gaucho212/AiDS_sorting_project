\documentclass{article}
%Tworzenie wykresów bezpośrednio w LaTeXu
\usepackage{pgfplots}
%Pozwala dodać opcję H dlo figur, dzięki czemu mamy absolutną kontrolę gdzie się mają pojawić
\usepackage{float}
%Ustawianie marginesu 1 cal lub 0.5 cala to najlepsza opcja. Standardowe 1.5 cala nadaje się dla książek ale dla sprawozdań już mniej
\usepackage[margin=1in]{geometry}
\pgfplotsset{
	width=8cm, height=7cm, ymin=0, xmin=2047, xmax=32768, grid=both, %wymiary osi
	xticklabel style={rotate=45, anchor=near xticklabel},  %liczby na osi X pod kątem 45*
	x label style={at={(axis description cs:0.5,-0.03)}}, %legenda osi X wyśrodkowana, lekko obniżona
	y label style={at={(axis description cs:-0.1,0.4)}}
}
\begin{document}
\section*{4 WYKRESY}

	%Wykresy, obrazy itd, warto opakowywać w `figure`, dzięki czemu można dodać Caption\Opis do figury, jak i `label` dzięki któremu później można odwoływać się do figur np. ``
\iffalse
	\begin{figure}[H]
		\centering
		\label{fig:enter-label}
		%Tworzę "pojemnik" na wykres `pgfplots` tak żeby automatycznie wyskalował mi wygenerowany wykres na szerokość strony. 
		\noindent\resizebox{\textwidth}{!}{
			\begin{tikzpicture}[scale=0.5]
				\begin{axis}[
					font=\scriptsize,
					name=plotA, anchor=left of south west,
					title={Złożoność Obliczeniowa Algorytmy Szybkie (decreasing array)}, 
					xlabel={Rozmiar instancji}, ylabel={Czas(ms)}, 
					legend pos=north west,
					legend style={font=\scriptsize, row sep=-2pt},
					xmode = log, log basis x={2},
					every axis plot post/.style={red!75!black, very thick},
					/tikz/plot label/.style={black, anchor=west}
					]
					\addplot[red, dashed, smooth] table[x=InputSize,y=Time,meta=Algorithm,col sep=comma] {insertion_sort_decreasing_array.txt};
					\addplot[red, dotted, smooth] table[x=InputSize,y=Time,meta=Algorithm,col sep=comma] {quick_sort_left_pivot_decreasing_array.txt};
					\legend{Insertion Sort, Quick Sort left}
				\end{axis}
			\end{tikzpicture}
			}
		\end{figure}
\fi
	\begin{figure}[H]
		\centering
		\label{fig:enter-label}
		%Tworzę "pojemnik" na wykres `pgfplots` tak żeby automatycznie wyskalował mi wygenerowany wykres na szerokość strony. 
		\noindent\resizebox{\textwidth}{!}{
			\begin{tikzpicture}
				\begin{axis}[%
					name=plotA, anchor=left of south west,
					title={Heap Sort A Shaped Array}, 
					xlabel={Rozmiar instancji}, ylabel={Czas(ms)}, 
					legend pos=north west,
					xmode = log, log basis x={2},
					every axis plot post/.style={red!75!black, very thick},
					/tikz/plot label/.style={black, anchor=west}
					]
					\addplot[red, dashed, smooth] table[x=InputSize,y=Time,meta=Algorithm,col sep=comma] {heap_sort_a_shaped_array.txt};
				\end{axis}
				\begin{axis}[%
					title={Insertion Sort A Shaped Array}, 
					name=plotB, at=(plotA.right of south east), 
					anchor=left of south west,
					xlabel={Rozmiar instancji}, ylabel={Czas(ms)}, 
					legend pos=north west,
					xmode = log, log basis x={2}, %Ustawiam oś x na logarytmiczną (log2)
					every axis plot post/.style={red!75!black, very thick},
					/tikz/plot label/.style={black, anchor=west}
					]
					\addplot[red, dashed, smooth] table[x=InputSize,y=Time,meta=Algorithm,col sep=comma] {insertion_sort_a_shaped_array.txt};
				\end{axis}
				\begin{axis}[%
					title={Quick Sort A Shaped Array}, 
					name=plotD, at=(plotB.below south west), 
					anchor=above north west,
					xlabel={Rozmiar instancji}, ylabel={Czas(ms)}, 
					legend pos=north west,
					xmode = log, log basis x={2},
					every axis plot post/.style={red!75!black, very thick},
					/tikz/plot label/.style={black, anchor=west}
					]
					\addplot[red, dashed, smooth] table[x=InputSize,y=Time,meta=Algorithm,col sep=comma] {quick_sort_left_pivot_a_shaped_array.txt};
					\addplot[red, dotted, smooth] table[x=InputSize,y=Time,meta=Algorithm,col sep=comma] {quick_sort_random_pivot_a_shaped_array.txt};
					\legend{Left Pivot, Random Pivot}
				\end{axis}
				\begin{axis}[%
					title={Selection Sort A Shaped Array},
					name=plotC, at=(plotD.left of south west), 
					anchor=right of south east,
					xlabel={Rozmiar instancji}, ylabel={Czas(ms)}, 
					legend pos=north west,
					xmode = log, log basis x={2},
					every axis plot post/.style={red!75!black, very thick},
					/tikz/plot label/.style={black, anchor=west}
					]
					\addplot[red, dashed, smooth] table[x=InputSize,y=Time,meta=Algorithm,col sep=comma] {selection_sort_a_shaped_array.txt};
				\end{axis}
				\begin{axis}[
					title={Shell Sort Sedgewick A shaped Array},
					title style={yshift=5pt},
					name=plotE, at=(plotD.below south west),
					%yshift=-0.8cm,
					xshift=2.1cm,
					anchor=above north east,
				%	yticklabel style={xshift=-1cm},
					xlabel={Rozmiar instancji}, ylabel={Czas(ms)},
					xmode = log, log basis x={2},
					every axis plot post/.style={red!75!black, very thick},
					/tikz/plot label/.style={black, anchor=north}
					]
					\addplot[red, dashed, smooth] table[x=InputSize,y=Time,meta=Algorithm,col sep=comma] {shell_sort_sedgewick_a_shaped_array.txt};
				\end{axis}
			\end{tikzpicture}
		}
	\end{figure}
\end{document}