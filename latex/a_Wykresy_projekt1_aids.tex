\label{wykresy}
\documentclass{article}
%Tworzenie wykresów bezpośrednio w LaTeXu
\usepackage{pgfplots}
%Pozwala dodać opcję H dlo figur, dzięki czemu mamy absolutną kontrolę gdzie się mają pojawić
\usepackage{float}
%Ustawianie marginesu 1 cal lub 0.5 cala to najlepsza opcja. Standardowe 1.5 cala nadaje się dla książek ale dla sprawozdań już mniej
\usepackage[margin=1in]{geometry}
\pgfplotsset{
	width=6cm, height=3cm, ymin=0, xmin=2047, xmax=32768, grid=both, %wymiary osi
	xticklabel style={rotate=45, anchor=near xticklabel},  %liczby na osi X pod kątem 45*
	x label style={at={(axis description cs:0.5,-0.5)}}, %legenda osi X wyśrodkowana, lekko obniżona
}
\begin{document}
	zawartość...%Wykresy, obrazy itd, warto opakowywać w `figure`, dzięki czemu można dodać Caption\Opis do figury, jak i `label` dzięki któremu później można odwoływać się do figur np. ``
	\begin{figure}[H]
		\centering
		
		%Tworzę "pojemnik" na wykres `pgfplots` tak żeby automatycznie wyskalował mi wygenerowany wykres na szerokość strony. 
		\noindent\resizebox{\textwidth}{!}{
			\begin{tikzpicture}
				\begin{axis}[%
					name=plotA, anchor=left of south west,
					width=6cm, height=3cm,%Nadpisuję rozmiar wykresu
					xlabel={ Rozmiar instancji}, ylabel={Czas (s)},
					%tick label style={font=\small}, % Zmniejszenie napisów na osiach
					%legend style={font=\small} ,legend pos=north west,% Zmniejszen 
					title={Złożoność Obliczeniowa Algorytmy Szybkie (plotA)}, 
					
					xmode = log, log basis x={2},
					every axis plot post/.style={red!75!black, very thick},
					/tikz/plot label/.style={black, anchor=west}
					]
					\addplot[red, dashed, smooth] table[x=InputSize,y=Time,meta=Algorithm,col sep=comma] {heap_sort_a_shaped_array.txt};
					\addplot[red, dotted, smooth] table[x=InputSize,y=Time,meta=Algorithm,col sep=comma] {heap_sort_constant_array.txt};
					\legend{Heap Sort, Quick Sort}
				\end{axis}
				
			\end{tikzpicture}
		}
		\caption{Caption}
		\label{fig:enter-label}
	\end{figure}
\end{document}