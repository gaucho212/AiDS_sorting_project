\documentclass{article}

\newcommand\reportsubject  {Laboratoria Algorytmów i Struktur Danych}
\newcommand\reporttitle    {Algorytmy Sortujące}
\newcommand\reportsubtitle {Sprawozdanie nr 1}
\newcommand\grouptutor     {Dominik Piotr Witczak}
\newcommand\teamduo		   {Marcin Jakubowski(164108), Adam Woźiński(164119)}
% Useful packages, sorted so packages of similar functionality are grouped together. Not all are essential to make the document work, however an effort was made to make this list as minimalistic as possible. Feel free to add your own!

% Essential for making this template work are graphicx, float, tabularx, tabu, tocbibind, titlesec, fancyhdr, xcolor and tikz. 

% Not essential, but you will have to debug the document a little bit when removing them are amsmath, amsthm, amssymb, amsfonts, caption, subcaption, appendix, enumitem, hyperref and cleveref.

% inputenc, lipsum, booktabs, geometry and microtype are not required, but nice to have.

\usepackage[T1]{fontenc}
\usepackage[polish]{babel}
\usepackage[utf8]{inputenc}
\usepackage{amsmath, amsthm, amssymb, amsfonts} % Nicer mathematical typesetting
\usepackage{multicol}
\usepackage{lipsum} % Creates dummy text lorem ipsum to showcase typsetting 
\usepackage{algorithm}
\usepackage{algpseudocode}
\usepackage{amsmath}
\usepackage{graphicx} % Allows the use of \begin{figure} and \includegraphics
\usepackage{float} % Useful for specifying the location of a figure ([H] for ex.)
\usepackage{caption} % Adds additional customization for (figure) captions
\usepackage{subcaption} % Needed to create sub-figures

\usepackage[nottoc,numbib]{tocbibind} % Automatically adds bibliography to ToC
\usepackage{titlesec} % Used to create custom section and subsection titles
\usepackage{titletoc} % Used to create a custom ToC
\usepackage{appendix} % Any chapter after \appendix is given a letter as index
\usepackage{fancyhdr} % Adds customization for headers and footers
\usepackage{hyperref} % Allows links and makes references and the ToC clickable
\usepackage[noabbrev, capitalise]{cleveref} % Easier referencing using \cref{<label>} instead of \ref{}

\usepackage{xcolor} % Predefines additional colors and allows user defined colors
\usepackage{tikz} % Useful for drawing images, used for creating the frontpage
\usetikzlibrary{positioning} % Additional library for relative positioning 
\usetikzlibrary{calc} % Additional library for calculating within tikz
\usetikzlibrary{shapes.geometric}
\usetikzlibrary{decorations.text}

% Defines a command used by tikz to calculate some coordinates for the front-page
\makeatletter
\newcommand{\gettikzxy}[3]{%
  \tikz@scan@one@point\pgfutil@firstofone#1\relax
  \edef#2{\the\pgf@x}%
  \edef#3{\the\pgf@y}%
}
\makeatother

% For terminal
\usepackage{fancyvrb,minted,tcolorbox}
\tcbuselibrary{skins,breakable,breakable}
\tcbuselibrary{minted}

% From previous

%Tworzenie wykresów bezpośrednio w LaTeXu
\usepackage{pgfplots}
%Pozwala dodać opcję H dlo figur, dzięki czemu mamy absolutną kontrolę gdzie się mają pojawić
\usepackage{float}
%Ustawianie marginesu 1 cal lub 0.5 cala to najlepsza opcja. Standardowe 1.5 cala nadaje się dla książek ale dla sprawozdań już mniej
\usepackage[margin=1in]{geometry}
\pgfplotsset{
	width=8cm, height=7cm, ymin=0, xmin=2047, xmax=32768, grid=both, %wymiary osi
	xticklabel style={rotate=45, anchor=near xticklabel},  %liczby na osi X pod kątem 45*
	x label style={at={(axis description cs:0.5,-0.03)}}, %legenda osi X wyśrodkowana, lekko obniżona
	y label style={at={(axis description cs:-0.1,0.4)}}
} % Loads in the preamble 
\newcommand\reporttitle{Konspekt}
\newcommand\reportsubtitle{course code, name - Qx (202x)}
\definecolor{PUT-Blue}{HTML}{00618E}
\definecolor{Light-Gray}{HTML}{F5F3F1}
\definecolor{Gray}{HTML}{A5A29D}
\definecolor{Medium-Gray}{HTML}{52514F}
\definecolor{Dark-Gray}{HTML}{21201F}

% Change bullet style for level 1, 2 and 3 respectively for enumerate
\renewcommand{\labelenumi}{\textbf{\textcolor{PUT-Blue}{\arabic*.}}}% level 1
\renewcommand{\labelenumii}{\textbf{\textcolor{PUT-Blue}{[\alph*]}}}% level 2
\renewcommand{\labelenumiii}{\textbf{\textcolor{PUT-Blue}{\roman*.}}}% level 3

\renewcommand{\labelitemi}{\textbf{\textcolor{PUT-Blue}{$\circ$}}}% level 1


\newcommand{\putbf}[1]{\textbf{\textcolor{PUT-Blue}{#1}}}

% Formats section, subsection and subsubsection titles respectively 
\titleformat{\section}{\sffamily\color{PUT-Blue}\Large\bfseries}{\thesection\enskip\color{gray}\textbar\enskip}{0cm}{} % Formats section titles

\titleformat{\subsection}{\sffamily\color{PUT-Blue}\large\bfseries}{\thesubsection\enskip\color{gray}\textbar\enskip}{0cm}{} % Formats subsection titles

\titleformat{\subsubsection}{\sffamily\color{PUT-Blue}\bfseries}{\thesubsubsection\enskip\color{gray}\textbar\enskip}{0cm}{} % Formats subsubsection titles

% Removes indent when starting a new paragraph
\setlength\parindent{0pt}

\definecolor{terminalColor}{RGB}{38,50,56}
\definecolor{Button1}{RGB}{254,94,86}
\definecolor{Button2}{RGB}{254,188,45}
\definecolor{Button3}{RGB}{38,202,59} % Loads in the preamble 

\begin{document}

\begin{titlepage}
	
	\begin{tikzpicture}[remember picture,overlay]
		% Default apex angle 30 degrees
		\node(left-triagle)[isosceles triangle,
		isosceles triangle apex angle=90,
		fill=Light-Gray,
		minimum size =0.6\textheight] (T.west) at (current page.north west){};
		
		\node(bottom-triagle)[isosceles triangle, name=bottomtriangle,
		isosceles triangle apex angle=90, rotate=90,
		fill=Light-Gray,
		minimum height=40.05cm] () at ([xshift=-9cm]current page.south east){};
		
		\node(bottom-rectangle)[rectangle,
		fill=Light-Gray, minimum height=47cm, minimum width=18cm] () at (current page.south east)
		{};
		
		\node[inner sep=0pt] (logo) at ([xshift=2.5cm, yshift=-2.5cm]current page.north west)
		{\includegraphics[width= 0.25\textwidth]{Figures/PP-PUT-LOGO.png}};
		
		\node[text width = 0.8\textwidth](subject) at (16,-4){\sffamily\Large \reportsubject};
		\node[text width = 0.8\textwidth, yshift = 0.75cm, xshift= -1.15cm, below = of subject](subtitle){\textcolor{PUT-Blue}{\sffamily\Large \reportsubtitle}};
		\node[text width = 0.8\textwidth, yshift = 0.75cm, xshift= -1.15cm, below = of subtitle](title) {\textcolor{PUT-Blue}{\sffamily\Huge\reporttitle}};
		\node[text width = 0.8\textwidth, yshift = 0.75cm, xshift= -1.25cm, below = of title](tutor){\sffamily\Large Prowadzący: \putbf{\grouptutor} };
		\node[text width = 0.8\textwidth, yshift = 0.75cm, xshift= -1.25cm, below = of tutor](names){\sffamily\normalsize Autorzy: \putbf{\teamduo} };
		
		
		
		
		\node[inner sep=0pt, anchor=west] (logo2) at ([xshift=1.2cm, yshift=2.5cm]current page.south west)
		{\includegraphics[width= 0.6\textwidth]{Figures/PP-PUT-WORD}};
		
	\end{tikzpicture}
	
\end{titlepage}
\newpage
	\section{Wprowadzenie}
	
	Celem projektu było stworzenie programu umożliwiającego użytkownikowi testowanie i porównywanie różnych algorytmów sortujących. Program został napisany w języku Python i zawiera implementacje kilku popularnych algorytmów sortowania, takich jak:
	
	\begin{itemize}
		\item \textbf{Insertion Sort} – sortowanie przez wstawianie,
		\item \textbf{Shell Sort} – sortowanie z przyrostami Sedgewicka,
		\item \textbf{Selection Sort} – sortowanie przez wybieranie,
		\item \textbf{Heap Sort} – sortowanie kopcowe,
		\item \textbf{Quick Sort} – sortowanie szybkie z różnymi strategiami wyboru pivota (lewy pivot i losowy pivot).
	\end{itemize}
	
	Każdy z algorytmów został zaimplementowany jako osobna funkcja, a ich działanie opiera się na kopiowaniu danych wejściowych, aby uniknąć modyfikacji oryginalnych danych. Program pozwala użytkownikowi wybrać jeden z dostępnych algorytmów sortowania i podać dane do posortowania w formie listy liczb.
	
	Dodatkowo, program mierzy czas działania wybranego algorytmu, co pozwala na ocenę jego wydajności w zależności od rozmiaru i charakterystyki danych wejściowych. W menu programu użytkownik może wybrać algorytm sortowania, a następnie wprowadzić dane ręcznie lub za pomocą standardowego wejścia.
	
	Projekt został zaprojektowany w sposób modularny, co ułatwia dodawanie nowych algorytmów sortowania w przyszłości. Dzięki temu użytkownik może w prosty sposób eksperymentować z różnymi metodami sortowania i analizować ich efektywność.
\newpage
	
	\section{Porównywanie czasów wykonywania}
	\section*{WYKRESY DLA DANYCH TYPU A SHAPED ARRAY }
	%Wykresy, obrazy itd, warto opakowywać w `figure`, dzięki czemu można dodać Caption\Opis do figury, jak i `label` dzięki któremu później można odwoływać się do figur np. ``
	\begin{figure}[H]
		\centering
		\label{fig:enter-label}
		%Tworzę "pojemnik" na wykres `pgfplots` tak żeby automatycznie wyskalował mi wygenerowany wykres na szerokość strony. 
		\noindent\resizebox{\textwidth}{!}{
			\begin{tikzpicture}
				\begin{axis}[%
					name=plotA, anchor=left of south west,
					title={Heap Sort A Shaped Array}, 
					xlabel={Rozmiar instancji}, ylabel={Czas(s)}, 
					legend pos=north west,
					xmode = log, log basis x={2},
					every axis plot post/.style={red!75!black, very thick},
					/tikz/plot label/.style={black, anchor=west}
					]
					\addplot[red, dashed, smooth] table[x=InputSize,y=Time,meta=Algorithm,col sep=comma] {heap_sort_a_shaped_array.txt};
				\end{axis}
				\begin{axis}[%
					title={Insertion Sort A Shaped Array}, 
					name=plotB, at=(plotA.right of south east), 
					anchor=left of south west,
					xlabel={Rozmiar instancji}, ylabel={Czas(s)}, 
					legend pos=north west,
					xmode = log, log basis x={2}, %Ustawiam oś x na logarytmiczną (log2)
					every axis plot post/.style={red!75!black, very thick},
					/tikz/plot label/.style={black, anchor=west}
					]
					\addplot[red, dashed, smooth] table[x=InputSize,y=Time,meta=Algorithm,col sep=comma] {insertion_sort_a_shaped_array.txt};
				\end{axis}
				\begin{axis}[%
					title={Quick Sort A Shaped Array}, 
					name=plotD, at=(plotB.below south west), 
					anchor=above north west,
					xlabel={Rozmiar instancji}, ylabel={Czas(s)}, 
					legend pos=north west,
					xmode = log, log basis x={2},
					every axis plot post/.style={red!75!black, very thick},
					/tikz/plot label/.style={black, anchor=west}
					]
					\addplot[red, dashed, smooth] table[x=InputSize,y=Time,meta=Algorithm,col sep=comma] {quick_sort_left_pivot_a_shaped_array.txt};
					\addplot[red, dotted, smooth] table[x=InputSize,y=Time,meta=Algorithm,col sep=comma] {quick_sort_random_pivot_a_shaped_array.txt};
					\legend{Left Pivot, Random Pivot}
				\end{axis}
				\begin{axis}[%
					title={Selection Sort A Shaped Array},
					name=plotC, at=(plotD.left of south west), 
					anchor=right of south east,
					xlabel={Rozmiar instancji}, ylabel={Czas(s)}, 
					legend pos=north west,
					xmode = log, log basis x={2},
					every axis plot post/.style={red!75!black, very thick},
					/tikz/plot label/.style={black, anchor=west}
					]
					\addplot[red, dashed, smooth] table[x=InputSize,y=Time,meta=Algorithm,col sep=comma] {selection_sort_a_shaped_array.txt};
				\end{axis}
				\begin{axis}[
					title={Shell Sort Sedgewick A Shaped Array},
					title style={yshift=5pt},
					name=plotE, at=(plotD.below south west),
					%yshift=-0.8cm,
					xshift=2.1cm,
					anchor=above north east,
					%	yticklabel style={xshift=-1cm},
					xlabel={Rozmiar instancji}, ylabel={Czas(s)},
					xmode = log, log basis x={2},
					every axis plot post/.style={red!75!black, very thick},
					/tikz/plot label/.style={black, anchor=north}
					]
					\addplot[red, dashed, smooth] table[x=InputSize,y=Time,meta=Algorithm,col sep=comma] {shell_sort_sedgewick_a_shaped_array.txt};
				\end{axis}
			\end{tikzpicture}
		}
	\end{figure}
	\newpage
	\section*{WYKRESY DLA DANYCH TYPU CONSTANT ARRAY }
	%Wykresy, obrazy itd, warto opakowywać w `figure`, dzięki czemu można dodać Caption\Opis do figury, jak i `label` dzięki któremu później można odwoływać się do figur np. ``
	\begin{figure}[H]
		\centering
		\label{fig:enter-label}
		%Tworzę "pojemnik" na wykres `pgfplots` tak żeby automatycznie wyskalował mi wygenerowany wykres na szerokość strony. 
		\noindent\resizebox{\textwidth}{!}{
			\begin{tikzpicture}
				\begin{axis}[%
					name=plotA, anchor=left of south west,
					title={Heap Sort Constant Array}, 
					xlabel={Rozmiar instancji}, ylabel={Czas(s)}, 
					legend pos=north west,
					xmode = log, log basis x={2},
					every axis plot post/.style={red!75!black, very thick},
					/tikz/plot label/.style={black, anchor=west}
					]
					\addplot[red, dashed, smooth] table[x=InputSize,y=Time,meta=Algorithm,col sep=comma] {heap_sort_constant_array.txt};
				\end{axis}
				\begin{axis}[%
					title={Insertion Sort Constant Array}, 
					name=plotB, at=(plotA.right of south east), 
					anchor=left of south west,
					xlabel={Rozmiar instancji}, ylabel={Czas(s)}, 
					legend pos=north west,
					xmode = log, log basis x={2}, %Ustawiam oś x na logarytmiczną (log2)
					every axis plot post/.style={red!75!black, very thick},
					/tikz/plot label/.style={black, anchor=west}
					]
					\addplot[red, dashed, smooth] table[x=InputSize,y=Time,meta=Algorithm,col sep=comma] {insertion_sort_constant_array.txt};
				\end{axis}
				\begin{axis}[%
					title={Quick Sort Constant Array}, 
					name=plotD, at=(plotB.below south west), 
					anchor=above north west,
					xlabel={Rozmiar instancji}, ylabel={Czas(s)}, 
					legend pos=north west,
					xmode = log, log basis x={2},
					every axis plot post/.style={red!75!black, very thick},
					/tikz/plot label/.style={black, anchor=west}
					]
					\addplot[red, dashed, smooth] table[x=InputSize,y=Time,meta=Algorithm,col sep=comma] {quick_sort_left_pivot_constant_array.txt};
					\addplot[red, dotted, smooth] table[x=InputSize,y=Time,meta=Algorithm,col sep=comma] {quick_sort_random_pivot_constant_array.txt};
					\legend{Left Pivot, Random Pivot}
				\end{axis}
				\begin{axis}[%
					title={Selection Sort Constant Array},
					name=plotC, at=(plotD.left of south west), 
					anchor=right of south east,
					xlabel={Rozmiar instancji}, ylabel={Czas(s)}, 
					legend pos=north west,
					xmode = log, log basis x={2},
					every axis plot post/.style={red!75!black, very thick},
					/tikz/plot label/.style={black, anchor=west}
					]
					\addplot[red, dashed, smooth] table[x=InputSize,y=Time,meta=Algorithm,col sep=comma] {selection_sort_constant_array.txt};
				\end{axis}
				\begin{axis}[
					title={Shell Sort Sedgewick Constant Array},
					title style={yshift=5pt},
					name=plotE, at=(plotD.below south west),
					%yshift=-0.8cm,
					xshift=2.1cm,
					anchor=above north east,
					%	yticklabel style={xshift=-1cm},
					xlabel={Rozmiar instancji}, ylabel={Czas(s)},
					xmode = log, log basis x={2},
					every axis plot post/.style={red!75!black, very thick},
					/tikz/plot label/.style={black, anchor=north}
					]
					\addplot[red, dashed, smooth] table[x=InputSize,y=Time,meta=Algorithm,col sep=comma] {shell_sort_sedgewick_constant_array.txt};
				\end{axis}
			\end{tikzpicture}
		}
	\end{figure}
	\newpage
	\section*{WYKRESY DLA DANYCH TYPU DECREASING ARRAY }
	%Wykresy, obrazy itd, warto opakowywać w `figure`, dzięki czemu można dodać Caption\Opis do figury, jak i `label` dzięki któremu później można odwoływać się do figur np. ``
	\begin{figure}[H]
		\centering
		\label{fig:enter-label}
		%Tworzę "pojemnik" na wykres `pgfplots` tak żeby automatycznie wyskalował mi wygenerowany wykres na szerokość strony. 
		\noindent\resizebox{\textwidth}{!}{
			\begin{tikzpicture}
				\begin{axis}[%
					name=plotA, anchor=left of south west,
					title={Heap Sort Decreasing Array}, 
					xlabel={Rozmiar instancji}, ylabel={Czas(s)}, 
					legend pos=north west,
					xmode = log, log basis x={2},
					every axis plot post/.style={red!75!black, very thick},
					/tikz/plot label/.style={black, anchor=west}
					]
					\addplot[red, dashed, smooth] table[x=InputSize,y=Time,meta=Algorithm,col sep=comma] {heap_sort_decreasing_array.txt};
				\end{axis}
				\begin{axis}[%
					title={Insertion Sort Decreasing Array}, 
					name=plotB, at=(plotA.right of south east), 
					anchor=left of south west,
					xlabel={Rozmiar instancji}, ylabel={Czas(s)}, 
					legend pos=north west,
					xmode = log, log basis x={2}, %Ustawiam oś x na logarytmiczną (log2)
					every axis plot post/.style={red!75!black, very thick},
					/tikz/plot label/.style={black, anchor=west}
					]
					\addplot[red, dashed, smooth] table[x=InputSize,y=Time,meta=Algorithm,col sep=comma] {insertion_sort_decreasing_array.txt};
				\end{axis}
				\begin{axis}[%
					title={Quick Sort Decreasing Array}, 
					name=plotD, at=(plotB.below south west), 
					anchor=above north west,
					xlabel={Rozmiar instancji}, ylabel={Czas(s)}, 
					legend pos=north west,
					xmode = log, log basis x={2},
					every axis plot post/.style={red!75!black, very thick},
					/tikz/plot label/.style={black, anchor=west}
					]
					\addplot[red, dashed, smooth] table[x=InputSize,y=Time,meta=Algorithm,col sep=comma] {quick_sort_left_pivot_decreasing_array.txt};
					\addplot[red, dotted, smooth] table[x=InputSize,y=Time,meta=Algorithm,col sep=comma] {quick_sort_random_pivot_decreasing_array.txt};
					\legend{Left Pivot, Random Pivot}
				\end{axis}
				\begin{axis}[%
					title={Selection Sort Decreasing Array},
					name=plotC, at=(plotD.left of south west), 
					anchor=right of south east,
					xlabel={Rozmiar instancji}, ylabel={Czas(s)}, 
					legend pos=north west,
					xmode = log, log basis x={2},
					every axis plot post/.style={red!75!black, very thick},
					/tikz/plot label/.style={black, anchor=west}
					]
					\addplot[red, dashed, smooth] table[x=InputSize,y=Time,meta=Algorithm,col sep=comma] {selection_sort_decreasing_array.txt};
				\end{axis}
				\begin{axis}[
					title={Shell Sort Sedgewick Decreasing Array},
					title style={yshift=5pt},
					name=plotE, at=(plotD.below south west),
					%yshift=-0.8cm,
					xshift=2.1cm,
					anchor=above north east,
					%	yticklabel style={xshift=-1cm},
					xlabel={Rozmiar instancji}, ylabel={Czas(s)},
					xmode = log, log basis x={2},
					every axis plot post/.style={red!75!black, very thick},
					/tikz/plot label/.style={black, anchor=north}
					]
					\addplot[red, dashed, smooth] table[x=InputSize,y=Time,meta=Algorithm,col sep=comma] {shell_sort_sedgewick_decreasing_array.txt};
				\end{axis}
			\end{tikzpicture}
		}
	\end{figure}
	\newpage
	\section*{WYKRESY DLA DANYCH TYPU INCREASING ARRAY }
	%Wykresy, obrazy itd, warto opakowywać w `figure`, dzięki czemu można dodać Caption\Opis do figury, jak i `label` dzięki któremu później można odwoływać się do figur np. ``
	\begin{figure}[H]
		\centering
		\label{fig:enter-label}
		%Tworzę "pojemnik" na wykres `pgfplots` tak żeby automatycznie wyskalował mi wygenerowany wykres na szerokość strony. 
		\noindent\resizebox{\textwidth}{!}{
			\begin{tikzpicture}
				\begin{axis}[%
					name=plotA, anchor=left of south west,
					title={Heap Sort Increasing Array}, 
					xlabel={Rozmiar instancji}, ylabel={Czas(s)}, 
					legend pos=north west,
					xmode = log, log basis x={2},
					every axis plot post/.style={red!75!black, very thick},
					/tikz/plot label/.style={black, anchor=west}
					]
					\addplot[red, dashed, smooth] table[x=InputSize,y=Time,meta=Algorithm,col sep=comma] {heap_sort_increasing_array.txt};
				\end{axis}
				\begin{axis}[%
					title={Insertion Sort Increasing Array}, 
					name=plotB, at=(plotA.right of south east), 
					anchor=left of south west,
					xlabel={Rozmiar instancji}, ylabel={Czas(s)}, 
					legend pos=north west,
					xmode = log, log basis x={2}, %Ustawiam oś x na logarytmiczną (log2)
					every axis plot post/.style={red!75!black, very thick},
					/tikz/plot label/.style={black, anchor=west}
					]
					\addplot[red, dashed, smooth] table[x=InputSize,y=Time,meta=Algorithm,col sep=comma] {insertion_sort_increasing_array.txt};
				\end{axis}
				\begin{axis}[%
					title={Quick Sort Increasing Array}, 
					name=plotD, at=(plotB.below south west), 
					anchor=above north west,
					xlabel={Rozmiar instancji}, ylabel={Czas(s)}, 
					legend pos=north west,
					xmode = log, log basis x={2},
					every axis plot post/.style={red!75!black, very thick},
					/tikz/plot label/.style={black, anchor=west}
					]
					\addplot[red, dashed, smooth] table[x=InputSize,y=Time,meta=Algorithm,col sep=comma] {quick_sort_left_pivot_increasing_array.txt};
					\addplot[red, dotted, smooth] table[x=InputSize,y=Time,meta=Algorithm,col sep=comma] {quick_sort_random_pivot_increasing_array.txt};
					\legend{Left Pivot, Random Pivot}
				\end{axis}
				\begin{axis}[%
					title={Selection Sort Increasing Array},
					name=plotC, at=(plotD.left of south west), 
					anchor=right of south east,
					xlabel={Rozmiar instancji}, ylabel={Czas(s)}, 
					legend pos=north west,
					xmode = log, log basis x={2},
					every axis plot post/.style={red!75!black, very thick},
					/tikz/plot label/.style={black, anchor=west}
					]
					\addplot[red, dashed, smooth] table[x=InputSize,y=Time,meta=Algorithm,col sep=comma] {selection_sort_increasing_array.txt};
				\end{axis}
				\begin{axis}[
					title={Shell Sort Sedgewick Increasing Array},
					title style={yshift=5pt},
					name=plotE, at=(plotD.below south west),
					%yshift=-0.8cm,
					xshift=2.1cm,
					anchor=above north east,
					%	yticklabel style={xshift=-1cm},
					xlabel={Rozmiar instancji}, ylabel={Czas(s)},
					xmode = log, log basis x={2},
					every axis plot post/.style={red!75!black, very thick},
					/tikz/plot label/.style={black, anchor=north}
					]
					\addplot[red, dashed, smooth] table[x=InputSize,y=Time,meta=Algorithm,col sep=comma] {shell_sort_sedgewick_increasing_array.txt};
				\end{axis}
			\end{tikzpicture}
		}
	\end{figure}
	\newpage
	\section*{WYKRESY DLA DANYCH TYPU RANDOM ARRAY }
	%Wykresy, obrazy itd, warto opakowywać w `figure`, dzięki czemu można dodać Caption\Opis do figury, jak i `label` dzięki któremu później można odwoływać się do figur np. ``
	\begin{figure}[H]
		\centering
		\label{fig:enter-label}
		%Tworzę "pojemnik" na wykres `pgfplots` tak żeby automatycznie wyskalował mi wygenerowany wykres na szerokość strony. 
		\noindent\resizebox{\textwidth}{!}{
			\begin{tikzpicture}
				\begin{axis}[%
					name=plotA, anchor=left of south west,
					title={Heap Sort Random Array}, 
					xlabel={Rozmiar instancji}, ylabel={Czas(s)}, 
					legend pos=north west,
					xmode = log, log basis x={2},
					every axis plot post/.style={red!75!black, very thick},
					/tikz/plot label/.style={black, anchor=west}
					]
					\addplot[red, dashed, smooth] table[x=InputSize,y=Time,meta=Algorithm,col sep=comma] {heap_sort_decreasing_array.txt};
				\end{axis}
				\begin{axis}[%
					title={Insertion Sort Random Array}, 
					name=plotB, at=(plotA.right of south east), 
					anchor=left of south west,
					xlabel={Rozmiar instancji}, ylabel={Czas(s)}, 
					legend pos=north west,
					xmode = log, log basis x={2}, %Ustawiam oś x na logarytmiczną (log2)
					every axis plot post/.style={red!75!black, very thick},
					/tikz/plot label/.style={black, anchor=west}
					]
					\addplot[red, dashed, smooth] table[x=InputSize,y=Time,meta=Algorithm,col sep=comma] {insertion_sort_random_array.txt};
				\end{axis}
				\begin{axis}[%
					title={Quick Sort Random Array}, 
					name=plotD, at=(plotB.below south west), 
					anchor=above north west,
					xlabel={Rozmiar instancji}, ylabel={Czas(s)}, 
					legend pos=north west,
					xmode = log, log basis x={2},
					every axis plot post/.style={red!75!black, very thick},
					/tikz/plot label/.style={black, anchor=west}
					]
					\addplot[red, dashed, smooth] table[x=InputSize,y=Time,meta=Algorithm,col sep=comma] {quick_sort_left_pivot_random_array.txt};
					\addplot[red, dotted, smooth] table[x=InputSize,y=Time,meta=Algorithm,col sep=comma] {quick_sort_random_pivot_random_array.txt};
					\legend{Left Pivot, Random Pivot}
				\end{axis}
				\begin{axis}[%
					title={Selection Sort Random Array},
					name=plotC, at=(plotD.left of south west), 
					anchor=right of south east,
					xlabel={Rozmiar instancji}, ylabel={Czas(s)}, 
					legend pos=north west,
					xmode = log, log basis x={2},
					every axis plot post/.style={red!75!black, very thick},
					/tikz/plot label/.style={black, anchor=west}
					]
					\addplot[red, dashed, smooth] table[x=InputSize,y=Time,meta=Algorithm,col sep=comma] {selection_sort_random_array.txt};
				\end{axis}
				\begin{axis}[
					title={Shell Sort Sedgewick Random Array},
					title style={yshift=5pt},
					name=plotE, at=(plotD.below south west),
					%yshift=-0.8cm,
					xshift=2.1cm,
					anchor=above north east,
					%	yticklabel style={xshift=-1cm},
					xlabel={Rozmiar instancji}, ylabel={Czas(s)},
					xmode = log, log basis x={2},
					every axis plot post/.style={red!75!black, very thick},
					/tikz/plot label/.style={black, anchor=north}
					]
					\addplot[red, dashed, smooth] table[x=InputSize,y=Time,meta=Algorithm,col sep=comma] {shell_sort_sedgewick_random_array.txt};
				\end{axis}
			\end{tikzpicture}
		}
	\end{figure}
	\newpage
	\section*{WYKRESY ZBIORCZE }
	%Wykresy, obrazy itd, warto opakowywać w `figure`, dzięki czemu można dodać Caption\Opis do figury, jak i `label` dzięki któremu później można odwoływać się do figur np. ``
	\begin{figure}[H]
		\centering
		\label{fig:enter-label}
		%Tworzę "pojemnik" na wykres `pgfplots` tak żeby automatycznie wyskalował mi wygenerowany wykres na szerokość strony. 
		\noindent\resizebox{\textwidth}{!}{
			\begin{tikzpicture}
				\begin{axis}[%
					name=plotA, anchor=left of south west,
					title={Heap Sort}, 
					xlabel={Rozmiar instancji}, ylabel={Czas(s)}, 
					legend pos=north west,
					xmode = log, log basis x={2},
					every axis plot post/.style={very thick},
					/tikz/plot label/.style={black, anchor=west}
					]
					\addplot[red, dashed, smooth] table[x=InputSize,y=Time,meta=Algorithm,col sep=comma] {heap_sort_a_shaped_array.txt};
					\addplot[color=blue, dashed, smooth] table[x=InputSize,y=Time,meta=Algorithm,col sep=comma] {heap_sort_constant_array.txt};
					\addplot[green, dotted, smooth] table[x=InputSize,y=Time,meta=Algorithm,col sep=comma] {heap_sort_decreasing_array.txt};
					\addplot[black, dotted, smooth] table[x=InputSize,y=Time,meta=Algorithm,col sep=comma] {heap_sort_increasing_array.txt};
					\addplot[yellow, dashed, smooth] table[x=InputSize,y=Time,meta=Algorithm,col sep=comma] {heap_sort_random_array.txt};
					\legend{A Shaped,Constant,Decreasing,Increasing,Random}
				\end{axis}
				\begin{axis}[%
					title={Insertion Sort}, 
					name=plotB, at=(plotA.right of south east), 
					anchor=left of south west,
					xlabel={Rozmiar instancji}, ylabel={Czas(s)}, 
					legend pos=north west,
					xmode = log, log basis x={2}, %Ustawiam oś x na logarytmiczną (log2)
					every axis plot post/.style={very thick},
					/tikz/plot label/.style={black, anchor=west}
					]
					\addplot[red, dashed, smooth] table[x=InputSize,y=Time,meta=Algorithm,col sep=comma] {insertion_sort_a_shaped_array.txt};
					\addplot[color=blue, dashed, smooth] table[x=InputSize,y=Time,meta=Algorithm,col sep=comma] {insertion_sort_constant_array.txt};
					\addplot[green, dotted, smooth] table[x=InputSize,y=Time,meta=Algorithm,col sep=comma] {insertion_sort_decreasing_array.txt};
					\addplot[black, dotted, smooth] table[x=InputSize,y=Time,meta=Algorithm,col sep=comma] {insertion_sort_increasing_array.txt};
					\addplot[yellow, dashed, smooth] table[x=InputSize,y=Time,meta=Algorithm,col sep=comma] {insertion_sort_random_array.txt};
					\legend{A Shaped,Constant,Decreasing,Increasing,Random}
				\end{axis}
				\begin{axis}[%
					title={Quick Sort Left Pivot}, 
					name=plotD, at=(plotB.below south west), 
					anchor=above north west,
					xlabel={Rozmiar instancji}, ylabel={Czas(s)}, 
					legend pos=north west,
					xmode = log, log basis x={2},
					every axis plot post/.style={very thick},
					/tikz/plot label/.style={black, anchor=west}
					]
					\addplot[red, dashed, smooth] table[x=InputSize,y=Time,meta=Algorithm,col sep=comma] {quick_sort_left_pivot_a_shaped_array.txt};
					\addplot[color=blue, dashed, smooth] table[x=InputSize,y=Time,meta=Algorithm,col sep=comma] {quick_sort_left_pivot_constant_array.txt};
					\addplot[green, dotted, smooth] table[x=InputSize,y=Time,meta=Algorithm,col sep=comma] {quick_sort_left_pivot_decreasing_array.txt};
					\addplot[black, dotted, smooth] table[x=InputSize,y=Time,meta=Algorithm,col sep=comma] {quick_sort_left_pivot_increasing_array.txt};
					\addplot[yellow, dashed, smooth] table[x=InputSize,y=Time,meta=Algorithm,col sep=comma] {quick_sort_left_pivot_random_array.txt};
					\legend{A Shaped,Constant,Decreasing,Increasing,Random}
				\end{axis}
				\begin{axis}[%
					title={Quick Sort Random Pivot},
					name=plotC, at=(plotD.left of south west), 
					anchor=right of south east,
					xlabel={Rozmiar instancji}, ylabel={Czas(s)}, 
					legend pos=north west,
					xmode = log, log basis x={2},
					every axis plot post/.style={very thick},
					/tikz/plot label/.style={black, anchor=west}
					]
					\addplot[red, dashed, smooth] table[x=InputSize,y=Time,meta=Algorithm,col sep=comma] {quick_sort_random_pivot_a_shaped_array.txt};
					\addplot[color=blue, dashed, smooth] table[x=InputSize,y=Time,meta=Algorithm,col sep=comma] {quick_sort_random_pivot_constant_array.txt};
					\addplot[green, dotted, smooth] table[x=InputSize,y=Time,meta=Algorithm,col sep=comma] {quick_sort_random_pivot_decreasing_array.txt};
					\addplot[black, dotted, smooth] table[x=InputSize,y=Time,meta=Algorithm,col sep=comma] {quick_sort_random_pivot_increasing_array.txt};
					\addplot[yellow, dashed, smooth] table[x=InputSize,y=Time,meta=Algorithm,col sep=comma] {quick_sort_random_pivot_random_array.txt};
					\legend{A Shaped,Constant,Decreasing,Increasing,Random}
				\end{axis}
				\begin{axis}[
					title={Selection Sort},
					title style={yshift=5pt},
					name=plotE, at=(plotD.below south west),
					%yshift=-0.8cm,
					%xshift=2.1cm,
					anchor=above north west,
					%	yticklabel style={xshift=-1cm},
					xlabel={Rozmiar instancji}, ylabel={Czas(s)},
					xmode = log, log basis x={2},
					every axis plot post/.style={very thick},
					/tikz/plot label/.style={black, anchor=north}
					]
					\addplot[red, dashed, smooth] table[x=InputSize,y=Time,meta=Algorithm,col sep=comma] {selection_sort_a_shaped_array.txt};
					\addplot[color=blue, dashed, smooth] table[x=InputSize,y=Time,meta=Algorithm,col sep=comma] {selection_sort_constant_array.txt};
					\addplot[green, dotted, smooth] table[x=InputSize,y=Time,meta=Algorithm,col sep=comma] {selection_sort_decreasing_array.txt};
					\addplot[black, dotted, smooth] table[x=InputSize,y=Time,meta=Algorithm,col sep=comma] {selection_sort_increasing_array.txt};
					\addplot[yellow, dashed, smooth] table[x=InputSize,y=Time,meta=Algorithm,col sep=comma] {selection_sort_random_array.txt};
					\legend{A Shaped,Constant,Decreasing,Increasing,Random}
				\end{axis}
				\begin{axis}[
					title={Shell Sort Sedgewick},
					title style={yshift=5pt},
					name=plotF, at=(plotE.left of south west),
					%yshift=-0.8cm,
					%xshift=2.1cm,
					anchor=right of south east,
					%	yticklabel style={xshift=-1cm},
					xlabel={Rozmiar instancji}, ylabel={Czas(s)},
					xmode = log, log basis x={2},
					every axis plot post/.style={very thick},
					/tikz/plot label/.style={black, anchor=north}
					]
					\addplot[red, dashed, smooth] table[x=InputSize,y=Time,meta=Algorithm,col sep=comma] {shell_sort_sedgewick_a_shaped_array.txt};
					\addplot[color=blue, dashed, smooth] table[x=InputSize,y=Time,meta=Algorithm,col sep=comma] {shell_sort_sedgewick_constant_array.txt};
					\addplot[green, dotted, smooth] table[x=InputSize,y=Time,meta=Algorithm,col sep=comma] {shell_sort_sedgewick_decreasing_array.txt};
					\addplot[black, dotted, smooth] table[x=InputSize,y=Time,meta=Algorithm,col sep=comma] {shell_sort_sedgewick_increasing_array.txt};
					\addplot[yellow, dashed, smooth] table[x=InputSize,y=Time,meta=Algorithm,col sep=comma] {shell_sort_sedgewick_random_array.txt};
					\legend{A Shaped,Constant,Decreasing,Increasing,Random}
				\end{axis}
			\end{tikzpicture}
		}
	\end{figure}
\newpage
	\section{Wnioski}
	
	Z analizy działania zaimplementowanych algorytmów sortowania wynika, że czas wykonania każdego z nich jest silnie uzależniony od charakterystyki danych wejściowych oraz rozmiaru listy. Algorytmy o złożoności kwadratowej, takie jak \textbf{Insertion Sort} czy \textbf{Selection Sort}, działają stosunkowo wolno dla dużych zbiorów danych, co czyni je nieefektywnymi w takich przypadkach. Z kolei algorytmy o złożoności \textbf{O(n log n)}, takie jak \textbf{Heap Sort} czy \textbf{Quick Sort}, wykazują znacznie lepszą wydajność, szczególnie dla dużych list.
	
	Warto również zauważyć, że wybór strategii wyboru pivota w \textbf{Quick Sort} ma istotny wpływ na czas działania algorytmu. W przypadku losowego wyboru pivota (\textbf{Quick Sort Random}) algorytm działa bardziej równomiernie w porównaniu do wyboru skrajnie lewego pivota (\textbf{Quick Sort Left}), który może prowadzić do nieoptymalnych podziałów w przypadku niekorzystnych danych wejściowych.
	
	Podsumowując, wybór odpowiedniego algorytmu sortowania powinien być uzależniony od charakterystyki danych oraz wymagań dotyczących wydajności. Algorytmy o złożoności \textbf{O(n log n)} są bardziej uniwersalne i efektywne dla dużych zbiorów danych, podczas gdy algorytmy o złożoności kwadratowej mogą być wystarczające dla małych list lub w sytuacjach, gdzie prostota implementacji jest priorytetem.
	
\end{document}