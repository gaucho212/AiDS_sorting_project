\newpage
  \begin{tikzpicture}[remember picture,overlay]
    % Default apex angle 30 degrees
    \node(bottom-rectangle)[rectangle,
        fill=Light-Gray, minimum height=5cm, minimum width=2\textwidth] () at (current page.north)
        {};
    
    \node(left-triagle)[isosceles triangle,
        isosceles triangle apex angle=90,
        fill=Light-Gray,
        minimum size =0.4\textheight] (T.west) at (current page.north west){};

    \node(bottom-rectangle)[rectangle,
        fill=Light-Gray, minimum height=5cm, minimum width=2\textwidth] () at (current page.south)
        {};

    \node(left-triagle)[isosceles triangle,
        isosceles triangle apex angle=90, rotate=90,
        fill=Light-Gray,
        minimum size =0.4\textheight] (T.west) at (current page.south east){};


    \node[inner sep=0pt, anchor=west] (logo) at ([xshift=1.2cm, yshift=-1.5cm]current page.north west)
    {\includegraphics[width= 0.4\textwidth]{Figures/PP-PUT-WORD.png}};

    \node[inner sep=0pt, anchor=center] (logo2) at ([xshift=-1.6cm, yshift=1.7cm]current page.south east)
    {\includegraphics[width= 2.2cm]{Figures/PP-PUT-WIIT-LOGO.png}};


    \draw [double distance=4mm,
           double=gray,
           draw opacity=0,
           rotate=150,
           anchor=center,
           postaction={
                decorate,
                decoration={
                      raise=-1ex,
                      text along path, 
                      reverse path,
                      text align={fit to path stretching spaces},
                      text={|\ttfamily\footnotesize\color{black}|Kierunek\space Informatyka\space |\ttfamily\footnotesize\color{gray}|Wydzial\space Informatyki\space i\space Telekomunikacji}
                }
           }
        ] (logo2.center) circle (1.4cm);
    
  \end{tikzpicture}
\section{Zadanie}
Zaimplementuj algorytm konstruowania \textbf{drzewa AVL} metodą \textbf{połowienia binarnego} oraz algorytm konstruowania \textbf{drzewa BST} z ciągu liczb. 


\begin{figure} [H]
    \noindent\resizebox{\textwidth}{!}{
\centering
\begin{subfigure}{0.4\textwidth}
    \centering
    \newtcblisting{TcblistingMintedTerminal}[0]{listing engine=minted,minted style=native,
    minted language=python,enhanced,
    minted options={fontsize=\tiny},  
    colback=terminalColor,colframe=terminalColor,listing only, title=\tikz {
        \node[circle,fill=Button1,inner sep=3pt] (c) at (0,0){};
        \node[circle,fill=Button2,inner sep=3pt] (c) at (0.5,0){};
        \node[circle,fill=Button3,inner sep=3pt] (c) at (1,0){};
    } ~~~~~~Terminal
}

\newenvironment{TikzTreeStyle}
{ % Create enviroment - a tikzpicture with my styling for nodes
    \begin{tikzpicture}
        [level distance=10mm,
        every node/.style={fill=red!60,circle,inner sep=1pt, minimum size=6mm},
        level 1/.style={sibling distance=20mm,nodes={fill=red!45}},
        level 2/.style={sibling distance=10mm,nodes={fill=red!30}},
        level 3/.style={sibling distance=5mm,nodes={fill=red!25}}]
}
{ % Close enviroment
    \end{tikzpicture}
}

\begin{TcblistingMintedTerminal}
>>>  ./program --tree AVL <<< 7 2 5 10 12 13 6 9
 nodes> 7
insert> 2 5 10 12 13 6 9
Sorted: 2, 5, 6, 9, 10, 12, 13
Median: 9
action>
\end{TcblistingMintedTerminal}

\begin{TikzTreeStyle}
  \node {9}
    child {node {5}
      child {node {2}}
      child {node {6}}
    }
    child {node {12}
      child {node {10}}
      child {node {13}}
    };
    \path[draw=none] (0,-2) -- (0,14mm); %Set tikzpicture height to 34mm
\end{TikzTreeStyle} 
    \caption{Tworzenie drzewa AVL}
    \label{fig:avl:create}
\end{subfigure}
\hfill
\begin{subfigure}{0.4\textwidth}
    \centering
    \newtcblisting{TcblistingMintedTerminal}[0]{listing engine=minted,minted style=native,
    minted language=python,enhanced,
    minted options={fontsize=\tiny},  
    colback=terminalColor,colframe=terminalColor,listing only, title=\tikz {
        \node[circle,fill=Button1,inner sep=3pt] (c) at (0,0){};
        \node[circle,fill=Button2,inner sep=3pt] (c) at (0.5,0){};
        \node[circle,fill=Button3,inner sep=3pt] (c) at (1,0){};
    } ~~~~~~Terminal
}

\newenvironment{TikzTreeStyle}
{ % Create enviroment - a tikzpicture with my styling for nodes
    \begin{tikzpicture}
        [level distance=10mm,
        every node/.style={fill=red!60,circle,inner sep=1pt, minimum size=6mm},
        level 1/.style={sibling distance=20mm,nodes={fill=red!45}},
        level 2/.style={sibling distance=10mm,nodes={fill=red!30}},
        level 3/.style={sibling distance=5mm,nodes={fill=red!25}}]
}
{ % Close enviroment
    \end{tikzpicture}
}

\begin{TcblistingMintedTerminal}
>>>  ./program --tree BST <<< 7 2 5 10 12 13 6 9
 nodes> 7
insert> 2 5 10 12 13 6 9
Inserting: 2, 5, 10, 12, 13, 6, 9
action>

\end{TcblistingMintedTerminal}

\begin{TikzTreeStyle}
\node {2}
child[missing]
child 
{ node {5}
  child
  { node {6}
    child[missing]
    child { node {9} } }
  child 
  { node {12}
    child { node {10} }
    child { node {13} } }
};
\path[draw=none] (0,-3) -- (0,4mm); %Set tikzpicture height to 34mm
\end{TikzTreeStyle} 
    \caption{Tworzenie drzewa BST}
    \label{fig:bst:create}
\end{subfigure}
}
\end{figure}

\section{Menu}
Kiedy oczekujesz od użytkownika wprowadzenia danych, wyświetl odpowiedni znak zachęty (np. `nodes>`) .

% ' nodes>' - do wprowadzenia ilości węzłów które użytkownik ma zamiar wpisać,

% `insert>` - do wporwadzenia węzłów do dodania,

% `remove>` - do wprowadzenia węzłów do usunięcia,

% `action>` - do wprowadzenia kolejnej akcji. 

Akcje wpisywane będą w postaci ciągów znaków (Print, Remove, Delte, Export, Rebalance, Exit)

\begin{tcblisting}{listing engine=minted,minted style=native,
    minted language=python,enhanced,
    minted options={fontsize=\tiny},  
    colback=terminalColor,colframe=terminalColor,listing only, title=\tikz {
        \node[circle,fill=Button1,inner sep=3pt] (c) at (0,0){};
        \node[circle,fill=Button2,inner sep=3pt] (c) at (0.5,0){};
        \node[circle,fill=Button3,inner sep=3pt] (c) at (1,0){};
    } ~~~~~~Terminal}
>>>  ./program --tree AVL <<< 2, 5, 10, 12, 13, 6, 9
 nodes> 7
insert> 2 5 10 12 13 6 9
action> Help
Help        Show this message
Print       Print the tree usin In-order, Pre-order, Post-order
Remove      Remove elements of the tree 
Delete      Delete whole tree
Export      Export the tree to tickzpicture
Rebalance   Rebalance the tree
Exit        Exits the program (same as ctrl+D)
action> Print
...prints the tree...
action> Remove
remove> 10 12
action> Rebalance
(ctrl + D)
Program exited with status: 0
>>>
\end{tcblisting}

Program powinien także obsługiwać podawanie tzw. heredoc oraz `przekierowanie z pliku`. Ale nie martwcie się, jeśli program da się wpisać z palca, to obsłuży on także `heredoc` oraz `przekierowanie z pliku`.


\begin{figure}[H]
\begin{subfigure}{0.48\textwidth}
    \centering
\begin{tcblisting}{listing engine=minted,minted style=native,
    minted language=python,enhanced,
    minted options={fontsize=\tiny},  
    colback=terminalColor,colframe=terminalColor,listing only, title=\tikz {
        \node[circle,fill=Button1,inner sep=3pt] (c) at (0,0){};
        \node[circle,fill=Button2,inner sep=3pt] (c) at (0.5,0){};
        \node[circle,fill=Button3,inner sep=3pt] (c) at (1,0){};
    } ~~~~~~Terminal}
>>>  ./program --tree AVL << EOD
> 7
> 2 5 10 12 13 6 9
> Print
> Exit
> EOD
\end{tcblisting}
    \caption{Heredoc}
    \label{fig:avl:create}
\end{subfigure}
\hfill
\begin{subfigure}{0.48\textwidth}
    \centering
\begin{tcblisting}{listing engine=minted,minted style=native,
    minted language=python,enhanced,
    minted options={fontsize=\tiny},  
    colback=terminalColor,colframe=terminalColor,listing only, title=\tikz {
        \node[circle,fill=Button1,inner sep=3pt] (c) at (0,0){};
        \node[circle,fill=Button2,inner sep=3pt] (c) at (0.5,0){};
        \node[circle,fill=Button3,inner sep=3pt] (c) at (1,0){};
    } ~~~~~~Terminal}
>>> more plik.txt
7
2 5 10 12 13 6 9
Print
Exit
>>>  ./program --tree AVL < plik.txt
\end{tcblisting}
    \caption{Przekierowanie z pliku}
    \label{fig:bst:create}
\end{subfigure}
\end{figure}
