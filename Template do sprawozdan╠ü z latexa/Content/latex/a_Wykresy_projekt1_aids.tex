
\newpage
  \begin{tikzpicture}[remember picture,overlay]
    % Default apex angle 30 degrees
    \node(bottom-rectangle)[rectangle,
        fill=Light-Gray, minimum height=5cm, minimum width=2\textwidth] () at (current page.north)
        {};
    
    \node(left-triagle)[isosceles triangle,
        isosceles triangle apex angle=90,
        fill=Light-Gray,
        minimum size =0.4\textheight] (T.west) at (current page.north west){};

    \node(bottom-rectangle)[rectangle,
        fill=Light-Gray, minimum height=5cm, minimum width=2\textwidth] () at (current page.south)
        {};

    \node(left-triagle)[isosceles triangle,
        isosceles triangle apex angle=90, rotate=90,
        fill=Light-Gray,
        minimum size =0.4\textheight] (T.west) at (current page.south east){};


    \node[inner sep=0pt, anchor=west] (logo) at ([xshift=1.2cm, yshift=-1.5cm]current page.north west)
    {\includegraphics[width= 0.4\textwidth]{Figures/PP-PUT-WORD.png}};

    \node[inner sep=0pt, anchor=center] (logo2) at ([xshift=-1.6cm, yshift=1.7cm]current page.south east)
    {\includegraphics[width= 2.2cm]{Figures/PP-PUT-WIIT-LOGO.png}};


    \draw [double distance=4mm,
           double=gray,
           draw opacity=0,
           rotate=150,
           anchor=center,
           postaction={
                decorate,
                decoration={
                      raise=-1ex,
                      text along path, 
                      reverse path,
                      text align={fit to path stretching spaces},
                      text={|\ttfamily\footnotesize\color{black}|Kierunek\space Informatyka\space |\ttfamily\footnotesize\color{gray}|Wydzial\space Informatyki\space i\space Telekomunikacji}
                }
           }
        ] (logo2.center) circle (1.4cm);
    
  \end{tikzpicture}
	\section*{WYKRESY DLA DANYCH TYPU A SHAPED ARRAY }
	%Wykresy, obrazy itd, warto opakowywać w `figure`, dzięki czemu można dodać Caption\Opis do figury, jak i `label` dzięki któremu później można odwoływać się do figur np. ``
	\begin{figure}[H]
		\centering
		\label{fig:enter-label}
		%Tworzę "pojemnik" na wykres `pgfplots` tak żeby automatycznie wyskalował mi wygenerowany wykres na szerokość strony. 
		\noindent\resizebox{\textwidth}{!}{
			\begin{tikzpicture}
				\begin{axis}[%
					name=plotA, anchor=left of south west,
					title={Heap Sort A Shaped Array}, 
					xlabel={Rozmiar instancji}, ylabel={Czas(s)}, 
					legend pos=north west,
					xmode = log, log basis x={2},
					every axis plot post/.style={red!75!black, very thick},
					/tikz/plot label/.style={black, anchor=west}
					]
					\addplot[red, dashed, smooth] table[x=InputSize,y=Time,meta=Algorithm,col sep=comma] {heap_sort_a_shaped_array.txt};
				\end{axis}
				\begin{axis}[%
					title={Insertion Sort A Shaped Array}, 
					name=plotB, at=(plotA.right of south east), 
					anchor=left of south west,
					xlabel={Rozmiar instancji}, ylabel={Czas(s)}, 
					legend pos=north west,
					xmode = log, log basis x={2}, %Ustawiam oś x na logarytmiczną (log2)
					every axis plot post/.style={red!75!black, very thick},
					/tikz/plot label/.style={black, anchor=west}
					]
					\addplot[red, dashed, smooth] table[x=InputSize,y=Time,meta=Algorithm,col sep=comma] {insertion_sort_a_shaped_array.txt};
				\end{axis}
				\begin{axis}[%
					title={Quick Sort A Shaped Array}, 
					name=plotD, at=(plotB.below south west), 
					anchor=above north west,
					xlabel={Rozmiar instancji}, ylabel={Czas(s)}, 
					legend pos=north west,
					xmode = log, log basis x={2},
					every axis plot post/.style={red!75!black, very thick},
					/tikz/plot label/.style={black, anchor=west}
					]
					\addplot[red, dashed, smooth] table[x=InputSize,y=Time,meta=Algorithm,col sep=comma] {quick_sort_left_pivot_a_shaped_array.txt};
					\addplot[red, dotted, smooth] table[x=InputSize,y=Time,meta=Algorithm,col sep=comma] {quick_sort_random_pivot_a_shaped_array.txt};
					\legend{Left Pivot, Random Pivot}
				\end{axis}
				\begin{axis}[%
					title={Selection Sort A Shaped Array},
					name=plotC, at=(plotD.left of south west), 
					anchor=right of south east,
					xlabel={Rozmiar instancji}, ylabel={Czas(s)}, 
					legend pos=north west,
					xmode = log, log basis x={2},
					every axis plot post/.style={red!75!black, very thick},
					/tikz/plot label/.style={black, anchor=west}
					]
					\addplot[red, dashed, smooth] table[x=InputSize,y=Time,meta=Algorithm,col sep=comma] {selection_sort_a_shaped_array.txt};
				\end{axis}
				\begin{axis}[
					title={Shell Sort Sedgewick A Shaped Array},
					title style={yshift=5pt},
					name=plotE, at=(plotD.below south west),
					%yshift=-0.8cm,
					xshift=2.1cm,
					anchor=above north east,
					%	yticklabel style={xshift=-1cm},
					xlabel={Rozmiar instancji}, ylabel={Czas(s)},
					xmode = log, log basis x={2},
					every axis plot post/.style={red!75!black, very thick},
					/tikz/plot label/.style={black, anchor=north}
					]
					\addplot[red, dashed, smooth] table[x=InputSize,y=Time,meta=Algorithm,col sep=comma] {shell_sort_sedgewick_a_shaped_array.txt};
				\end{axis}
			\end{tikzpicture}
		}
	\end{figure}
	\newpage
      \begin{tikzpicture}[remember picture,overlay]
    % Default apex angle 30 degrees
    \node(bottom-rectangle)[rectangle,
        fill=Light-Gray, minimum height=5cm, minimum width=2\textwidth] () at (current page.north)
        {};
    
    \node(left-triagle)[isosceles triangle,
        isosceles triangle apex angle=90,
        fill=Light-Gray,
        minimum size =0.4\textheight] (T.west) at (current page.north west){};

    \node(bottom-rectangle)[rectangle,
        fill=Light-Gray, minimum height=5cm, minimum width=2\textwidth] () at (current page.south)
        {};

    \node(left-triagle)[isosceles triangle,
        isosceles triangle apex angle=90, rotate=90,
        fill=Light-Gray,
        minimum size =0.4\textheight] (T.west) at (current page.south east){};


    \node[inner sep=0pt, anchor=west] (logo) at ([xshift=1.2cm, yshift=-1.5cm]current page.north west)
    {\includegraphics[width= 0.4\textwidth]{Figures/PP-PUT-WORD.png}};

    \node[inner sep=0pt, anchor=center] (logo2) at ([xshift=-1.6cm, yshift=1.7cm]current page.south east)
    {\includegraphics[width= 2.2cm]{Figures/PP-PUT-WIIT-LOGO.png}};


    \draw [double distance=4mm,
           double=gray,
           draw opacity=0,
           rotate=150,
           anchor=center,
           postaction={
                decorate,
                decoration={
                      raise=-1ex,
                      text along path, 
                      reverse path,
                      text align={fit to path stretching spaces},
                      text={|\ttfamily\footnotesize\color{black}|Kierunek\space Informatyka\space |\ttfamily\footnotesize\color{gray}|Wydzial\space Informatyki\space i\space Telekomunikacji}
                }
           }
        ] (logo2.center) circle (1.4cm);
    
  \end{tikzpicture}
	\section*{WYKRESY DLA DANYCH TYPU CONSTANT ARRAY }
	%Wykresy, obrazy itd, warto opakowywać w `figure`, dzięki czemu można dodać Caption\Opis do figury, jak i `label` dzięki któremu później można odwoływać się do figur np. ``
	\begin{figure}[H]
		\centering
		\label{fig:enter-label}
		%Tworzę "pojemnik" na wykres `pgfplots` tak żeby automatycznie wyskalował mi wygenerowany wykres na szerokość strony. 
		\noindent\resizebox{\textwidth}{!}{
			\begin{tikzpicture}
				\begin{axis}[%
					name=plotA, anchor=left of south west,
					title={Heap Sort Constant Array}, 
					xlabel={Rozmiar instancji}, ylabel={Czas(s)}, 
					legend pos=north west,
					xmode = log, log basis x={2},
					every axis plot post/.style={red!75!black, very thick},
					/tikz/plot label/.style={black, anchor=west}
					]
					\addplot[red, dashed, smooth] table[x=InputSize,y=Time,meta=Algorithm,col sep=comma] {heap_sort_constant_array.txt};
				\end{axis}
				\begin{axis}[%
					title={Insertion Sort Constant Array}, 
					name=plotB, at=(plotA.right of south east), 
					anchor=left of south west,
					xlabel={Rozmiar instancji}, ylabel={Czas(s)}, 
					legend pos=north west,
					xmode = log, log basis x={2}, %Ustawiam oś x na logarytmiczną (log2)
					every axis plot post/.style={red!75!black, very thick},
					/tikz/plot label/.style={black, anchor=west}
					]
					\addplot[red, dashed, smooth] table[x=InputSize,y=Time,meta=Algorithm,col sep=comma] {insertion_sort_constant_array.txt};
				\end{axis}
				\begin{axis}[%
					title={Quick Sort Constant Array}, 
					name=plotD, at=(plotB.below south west), 
					anchor=above north west,
					xlabel={Rozmiar instancji}, ylabel={Czas(s)}, 
					legend pos=north west,
					xmode = log, log basis x={2},
					every axis plot post/.style={red!75!black, very thick},
					/tikz/plot label/.style={black, anchor=west}
					]
					\addplot[red, dashed, smooth] table[x=InputSize,y=Time,meta=Algorithm,col sep=comma] {quick_sort_left_pivot_constant_array.txt};
					\addplot[red, dotted, smooth] table[x=InputSize,y=Time,meta=Algorithm,col sep=comma] {quick_sort_random_pivot_constant_array.txt};
					\legend{Left Pivot, Random Pivot}
				\end{axis}
				\begin{axis}[%
					title={Selection Sort Constant Array},
					name=plotC, at=(plotD.left of south west), 
					anchor=right of south east,
					xlabel={Rozmiar instancji}, ylabel={Czas(s)}, 
					legend pos=north west,
					xmode = log, log basis x={2},
					every axis plot post/.style={red!75!black, very thick},
					/tikz/plot label/.style={black, anchor=west}
					]
					\addplot[red, dashed, smooth] table[x=InputSize,y=Time,meta=Algorithm,col sep=comma] {selection_sort_constant_array.txt};
				\end{axis}
				\begin{axis}[
					title={Shell Sort Sedgewick Constant Array},
					title style={yshift=5pt},
					name=plotE, at=(plotD.below south west),
					%yshift=-0.8cm,
					xshift=2.1cm,
					anchor=above north east,
					%	yticklabel style={xshift=-1cm},
					xlabel={Rozmiar instancji}, ylabel={Czas(s)},
					xmode = log, log basis x={2},
					every axis plot post/.style={red!75!black, very thick},
					/tikz/plot label/.style={black, anchor=north}
					]
					\addplot[red, dashed, smooth] table[x=InputSize,y=Time,meta=Algorithm,col sep=comma] {shell_sort_sedgewick_constant_array.txt};
				\end{axis}
			\end{tikzpicture}
		}
	\end{figure}
	\newpage
      \begin{tikzpicture}[remember picture,overlay]
    % Default apex angle 30 degrees
    \node(bottom-rectangle)[rectangle,
        fill=Light-Gray, minimum height=5cm, minimum width=2\textwidth] () at (current page.north)
        {};
    
    \node(left-triagle)[isosceles triangle,
        isosceles triangle apex angle=90,
        fill=Light-Gray,
        minimum size =0.4\textheight] (T.west) at (current page.north west){};

    \node(bottom-rectangle)[rectangle,
        fill=Light-Gray, minimum height=5cm, minimum width=2\textwidth] () at (current page.south)
        {};

    \node(left-triagle)[isosceles triangle,
        isosceles triangle apex angle=90, rotate=90,
        fill=Light-Gray,
        minimum size =0.4\textheight] (T.west) at (current page.south east){};


    \node[inner sep=0pt, anchor=west] (logo) at ([xshift=1.2cm, yshift=-1.5cm]current page.north west)
    {\includegraphics[width= 0.4\textwidth]{Figures/PP-PUT-WORD.png}};

    \node[inner sep=0pt, anchor=center] (logo2) at ([xshift=-1.6cm, yshift=1.7cm]current page.south east)
    {\includegraphics[width= 2.2cm]{Figures/PP-PUT-WIIT-LOGO.png}};


    \draw [double distance=4mm,
           double=gray,
           draw opacity=0,
           rotate=150,
           anchor=center,
           postaction={
                decorate,
                decoration={
                      raise=-1ex,
                      text along path, 
                      reverse path,
                      text align={fit to path stretching spaces},
                      text={|\ttfamily\footnotesize\color{black}|Kierunek\space Informatyka\space |\ttfamily\footnotesize\color{gray}|Wydzial\space Informatyki\space i\space Telekomunikacji}
                }
           }
        ] (logo2.center) circle (1.4cm);
    
  \end{tikzpicture}
	\section*{WYKRESY DLA DANYCH TYPU DECREASING ARRAY }
	%Wykresy, obrazy itd, warto opakowywać w `figure`, dzięki czemu można dodać Caption\Opis do figury, jak i `label` dzięki któremu później można odwoływać się do figur np. ``
	\begin{figure}[H]
		\centering
		\label{fig:enter-label}
		%Tworzę "pojemnik" na wykres `pgfplots` tak żeby automatycznie wyskalował mi wygenerowany wykres na szerokość strony. 
		\noindent\resizebox{\textwidth}{!}{
			\begin{tikzpicture}
				\begin{axis}[%
					name=plotA, anchor=left of south west,
					title={Heap Sort Decreasing Array}, 
					xlabel={Rozmiar instancji}, ylabel={Czas(s)}, 
					legend pos=north west,
					xmode = log, log basis x={2},
					every axis plot post/.style={red!75!black, very thick},
					/tikz/plot label/.style={black, anchor=west}
					]
					\addplot[red, dashed, smooth] table[x=InputSize,y=Time,meta=Algorithm,col sep=comma] {heap_sort_decreasing_array.txt};
				\end{axis}
				\begin{axis}[%
					title={Insertion Sort Decreasing Array}, 
					name=plotB, at=(plotA.right of south east), 
					anchor=left of south west,
					xlabel={Rozmiar instancji}, ylabel={Czas(s)}, 
					legend pos=north west,
					xmode = log, log basis x={2}, %Ustawiam oś x na logarytmiczną (log2)
					every axis plot post/.style={red!75!black, very thick},
					/tikz/plot label/.style={black, anchor=west}
					]
					\addplot[red, dashed, smooth] table[x=InputSize,y=Time,meta=Algorithm,col sep=comma] {insertion_sort_decreasing_array.txt};
				\end{axis}
				\begin{axis}[%
					title={Quick Sort Decreasing Array}, 
					name=plotD, at=(plotB.below south west), 
					anchor=above north west,
					xlabel={Rozmiar instancji}, ylabel={Czas(s)}, 
					legend pos=north west,
					xmode = log, log basis x={2},
					every axis plot post/.style={red!75!black, very thick},
					/tikz/plot label/.style={black, anchor=west}
					]
					\addplot[red, dashed, smooth] table[x=InputSize,y=Time,meta=Algorithm,col sep=comma] {quick_sort_left_pivot_decreasing_array.txt};
					\addplot[red, dotted, smooth] table[x=InputSize,y=Time,meta=Algorithm,col sep=comma] {quick_sort_random_pivot_decreasing_array.txt};
					\legend{Left Pivot, Random Pivot}
				\end{axis}
				\begin{axis}[%
					title={Selection Sort Decreasing Array},
					name=plotC, at=(plotD.left of south west), 
					anchor=right of south east,
					xlabel={Rozmiar instancji}, ylabel={Czas(s)}, 
					legend pos=north west,
					xmode = log, log basis x={2},
					every axis plot post/.style={red!75!black, very thick},
					/tikz/plot label/.style={black, anchor=west}
					]
					\addplot[red, dashed, smooth] table[x=InputSize,y=Time,meta=Algorithm,col sep=comma] {selection_sort_decreasing_array.txt};
				\end{axis}
				\begin{axis}[
					title={Shell Sort Sedgewick Decreasing Array},
					title style={yshift=5pt},
					name=plotE, at=(plotD.below south west),
					%yshift=-0.8cm,
					xshift=2.1cm,
					anchor=above north east,
					%	yticklabel style={xshift=-1cm},
					xlabel={Rozmiar instancji}, ylabel={Czas(s)},
					xmode = log, log basis x={2},
					every axis plot post/.style={red!75!black, very thick},
					/tikz/plot label/.style={black, anchor=north}
					]
					\addplot[red, dashed, smooth] table[x=InputSize,y=Time,meta=Algorithm,col sep=comma] {shell_sort_sedgewick_decreasing_array.txt};
				\end{axis}
			\end{tikzpicture}
		}
	\end{figure}
	\newpage
      \begin{tikzpicture}[remember picture,overlay]
    % Default apex angle 30 degrees
    \node(bottom-rectangle)[rectangle,
        fill=Light-Gray, minimum height=5cm, minimum width=2\textwidth] () at (current page.north)
        {};
    
    \node(left-triagle)[isosceles triangle,
        isosceles triangle apex angle=90,
        fill=Light-Gray,
        minimum size =0.4\textheight] (T.west) at (current page.north west){};

    \node(bottom-rectangle)[rectangle,
        fill=Light-Gray, minimum height=5cm, minimum width=2\textwidth] () at (current page.south)
        {};

    \node(left-triagle)[isosceles triangle,
        isosceles triangle apex angle=90, rotate=90,
        fill=Light-Gray,
        minimum size =0.4\textheight] (T.west) at (current page.south east){};


    \node[inner sep=0pt, anchor=west] (logo) at ([xshift=1.2cm, yshift=-1.5cm]current page.north west)
    {\includegraphics[width= 0.4\textwidth]{Figures/PP-PUT-WORD.png}};

    \node[inner sep=0pt, anchor=center] (logo2) at ([xshift=-1.6cm, yshift=1.7cm]current page.south east)
    {\includegraphics[width= 2.2cm]{Figures/PP-PUT-WIIT-LOGO.png}};


    \draw [double distance=4mm,
           double=gray,
           draw opacity=0,
           rotate=150,
           anchor=center,
           postaction={
                decorate,
                decoration={
                      raise=-1ex,
                      text along path, 
                      reverse path,
                      text align={fit to path stretching spaces},
                      text={|\ttfamily\footnotesize\color{black}|Kierunek\space Informatyka\space |\ttfamily\footnotesize\color{gray}|Wydzial\space Informatyki\space i\space Telekomunikacji}
                }
           }
        ] (logo2.center) circle (1.4cm);
    
  \end{tikzpicture}
	\section*{WYKRESY DLA DANYCH TYPU INCREASING ARRAY }
	%Wykresy, obrazy itd, warto opakowywać w `figure`, dzięki czemu można dodać Caption\Opis do figury, jak i `label` dzięki któremu później można odwoływać się do figur np. ``
	\begin{figure}[H]
		\centering
		\label{fig:enter-label}
		%Tworzę "pojemnik" na wykres `pgfplots` tak żeby automatycznie wyskalował mi wygenerowany wykres na szerokość strony. 
		\noindent\resizebox{\textwidth}{!}{
			\begin{tikzpicture}
				\begin{axis}[%
					name=plotA, anchor=left of south west,
					title={Heap Sort Increasing Array}, 
					xlabel={Rozmiar instancji}, ylabel={Czas(s)}, 
					legend pos=north west,
					xmode = log, log basis x={2},
					every axis plot post/.style={red!75!black, very thick},
					/tikz/plot label/.style={black, anchor=west}
					]
					\addplot[red, dashed, smooth] table[x=InputSize,y=Time,meta=Algorithm,col sep=comma] {heap_sort_increasing_array.txt};
				\end{axis}
				\begin{axis}[%
					title={Insertion Sort Increasing Array}, 
					name=plotB, at=(plotA.right of south east), 
					anchor=left of south west,
					xlabel={Rozmiar instancji}, ylabel={Czas(s)}, 
					legend pos=north west,
					xmode = log, log basis x={2}, %Ustawiam oś x na logarytmiczną (log2)
					every axis plot post/.style={red!75!black, very thick},
					/tikz/plot label/.style={black, anchor=west}
					]
					\addplot[red, dashed, smooth] table[x=InputSize,y=Time,meta=Algorithm,col sep=comma] {insertion_sort_increasing_array.txt};
				\end{axis}
				\begin{axis}[%
					title={Quick Sort Increasing Array}, 
					name=plotD, at=(plotB.below south west), 
					anchor=above north west,
					xlabel={Rozmiar instancji}, ylabel={Czas(s)}, 
					legend pos=north west,
					xmode = log, log basis x={2},
					every axis plot post/.style={red!75!black, very thick},
					/tikz/plot label/.style={black, anchor=west}
					]
					\addplot[red, dashed, smooth] table[x=InputSize,y=Time,meta=Algorithm,col sep=comma] {quick_sort_left_pivot_increasing_array.txt};
					\addplot[red, dotted, smooth] table[x=InputSize,y=Time,meta=Algorithm,col sep=comma] {quick_sort_random_pivot_increasing_array.txt};
					\legend{Left Pivot, Random Pivot}
				\end{axis}
				\begin{axis}[%
					title={Selection Sort Increasing Array},
					name=plotC, at=(plotD.left of south west), 
					anchor=right of south east,
					xlabel={Rozmiar instancji}, ylabel={Czas(s)}, 
					legend pos=north west,
					xmode = log, log basis x={2},
					every axis plot post/.style={red!75!black, very thick},
					/tikz/plot label/.style={black, anchor=west}
					]
					\addplot[red, dashed, smooth] table[x=InputSize,y=Time,meta=Algorithm,col sep=comma] {selection_sort_increasing_array.txt};
				\end{axis}
				\begin{axis}[
					title={Shell Sort Sedgewick Increasing Array},
					title style={yshift=5pt},
					name=plotE, at=(plotD.below south west),
					%yshift=-0.8cm,
					xshift=2.1cm,
					anchor=above north east,
					%	yticklabel style={xshift=-1cm},
					xlabel={Rozmiar instancji}, ylabel={Czas(s)},
					xmode = log, log basis x={2},
					every axis plot post/.style={red!75!black, very thick},
					/tikz/plot label/.style={black, anchor=north}
					]
					\addplot[red, dashed, smooth] table[x=InputSize,y=Time,meta=Algorithm,col sep=comma] {shell_sort_sedgewick_increasing_array.txt};
				\end{axis}
			\end{tikzpicture}
		}
	\end{figure}
	\newpage
      \begin{tikzpicture}[remember picture,overlay]
    % Default apex angle 30 degrees
    \node(bottom-rectangle)[rectangle,
        fill=Light-Gray, minimum height=5cm, minimum width=2\textwidth] () at (current page.north)
        {};
    
    \node(left-triagle)[isosceles triangle,
        isosceles triangle apex angle=90,
        fill=Light-Gray,
        minimum size =0.4\textheight] (T.west) at (current page.north west){};

    \node(bottom-rectangle)[rectangle,
        fill=Light-Gray, minimum height=5cm, minimum width=2\textwidth] () at (current page.south)
        {};

    \node(left-triagle)[isosceles triangle,
        isosceles triangle apex angle=90, rotate=90,
        fill=Light-Gray,
        minimum size =0.4\textheight] (T.west) at (current page.south east){};


    \node[inner sep=0pt, anchor=west] (logo) at ([xshift=1.2cm, yshift=-1.5cm]current page.north west)
    {\includegraphics[width= 0.4\textwidth]{Figures/PP-PUT-WORD.png}};

    \node[inner sep=0pt, anchor=center] (logo2) at ([xshift=-1.6cm, yshift=1.7cm]current page.south east)
    {\includegraphics[width= 2.2cm]{Figures/PP-PUT-WIIT-LOGO.png}};


    \draw [double distance=4mm,
           double=gray,
           draw opacity=0,
           rotate=150,
           anchor=center,
           postaction={
                decorate,
                decoration={
                      raise=-1ex,
                      text along path, 
                      reverse path,
                      text align={fit to path stretching spaces},
                      text={|\ttfamily\footnotesize\color{black}|Kierunek\space Informatyka\space |\ttfamily\footnotesize\color{gray}|Wydzial\space Informatyki\space i\space Telekomunikacji}
                }
           }
        ] (logo2.center) circle (1.4cm);
    
  \end{tikzpicture}
	\section*{WYKRESY DLA DANYCH TYPU RANDOM ARRAY }
	%Wykresy, obrazy itd, warto opakowywać w `figure`, dzięki czemu można dodać Caption\Opis do figury, jak i `label` dzięki któremu później można odwoływać się do figur np. ``
	\begin{figure}[H]
		\centering
		\label{fig:enter-label}
		%Tworzę "pojemnik" na wykres `pgfplots` tak żeby automatycznie wyskalował mi wygenerowany wykres na szerokość strony. 
		\noindent\resizebox{\textwidth}{!}{
			\begin{tikzpicture}
				\begin{axis}[%
					name=plotA, anchor=left of south west,
					title={Heap Sort Random Array}, 
					xlabel={Rozmiar instancji}, ylabel={Czas(s)}, 
					legend pos=north west,
					xmode = log, log basis x={2},
					every axis plot post/.style={red!75!black, very thick},
					/tikz/plot label/.style={black, anchor=west}
					]
					\addplot[red, dashed, smooth] table[x=InputSize,y=Time,meta=Algorithm,col sep=comma] {heap_sort_decreasing_array.txt};
				\end{axis}
				\begin{axis}[%
					title={Insertion Sort Random Array}, 
					name=plotB, at=(plotA.right of south east), 
					anchor=left of south west,
					xlabel={Rozmiar instancji}, ylabel={Czas(s)}, 
					legend pos=north west,
					xmode = log, log basis x={2}, %Ustawiam oś x na logarytmiczną (log2)
					every axis plot post/.style={red!75!black, very thick},
					/tikz/plot label/.style={black, anchor=west}
					]
					\addplot[red, dashed, smooth] table[x=InputSize,y=Time,meta=Algorithm,col sep=comma] {insertion_sort_random_array.txt};
				\end{axis}
				\begin{axis}[%
					title={Quick Sort Random Array}, 
					name=plotD, at=(plotB.below south west), 
					anchor=above north west,
					xlabel={Rozmiar instancji}, ylabel={Czas(s)}, 
					legend pos=north west,
					xmode = log, log basis x={2},
					every axis plot post/.style={red!75!black, very thick},
					/tikz/plot label/.style={black, anchor=west}
					]
					\addplot[red, dashed, smooth] table[x=InputSize,y=Time,meta=Algorithm,col sep=comma] {quick_sort_left_pivot_random_array.txt};
					\addplot[red, dotted, smooth] table[x=InputSize,y=Time,meta=Algorithm,col sep=comma] {quick_sort_random_pivot_random_array.txt};
					\legend{Left Pivot, Random Pivot}
				\end{axis}
				\begin{axis}[%
					title={Selection Sort Random Array},
					name=plotC, at=(plotD.left of south west), 
					anchor=right of south east,
					xlabel={Rozmiar instancji}, ylabel={Czas(s)}, 
					legend pos=north west,
					xmode = log, log basis x={2},
					every axis plot post/.style={red!75!black, very thick},
					/tikz/plot label/.style={black, anchor=west}
					]
					\addplot[red, dashed, smooth] table[x=InputSize,y=Time,meta=Algorithm,col sep=comma] {selection_sort_random_array.txt};
				\end{axis}
				\begin{axis}[
					title={Shell Sort Sedgewick Random Array},
					title style={yshift=5pt},
					name=plotE, at=(plotD.below south west),
					%yshift=-0.8cm,
					xshift=2.1cm,
					anchor=above north east,
					%	yticklabel style={xshift=-1cm},
					xlabel={Rozmiar instancji}, ylabel={Czas(s)},
					xmode = log, log basis x={2},
					every axis plot post/.style={red!75!black, very thick},
					/tikz/plot label/.style={black, anchor=north}
					]
					\addplot[red, dashed, smooth] table[x=InputSize,y=Time,meta=Algorithm,col sep=comma] {shell_sort_sedgewick_random_array.txt};
				\end{axis}
			\end{tikzpicture}
		}
	\end{figure}
	\newpage
      \begin{tikzpicture}[remember picture,overlay]
    % Default apex angle 30 degrees
    \node(bottom-rectangle)[rectangle,
        fill=Light-Gray, minimum height=5cm, minimum width=2\textwidth] () at (current page.north)
        {};
    
    \node(left-triagle)[isosceles triangle,
        isosceles triangle apex angle=90,
        fill=Light-Gray,
        minimum size =0.4\textheight] (T.west) at (current page.north west){};

    \node(bottom-rectangle)[rectangle,
        fill=Light-Gray, minimum height=5cm, minimum width=2\textwidth] () at (current page.south)
        {};

    \node(left-triagle)[isosceles triangle,
        isosceles triangle apex angle=90, rotate=90,
        fill=Light-Gray,
        minimum size =0.4\textheight] (T.west) at (current page.south east){};


    \node[inner sep=0pt, anchor=west] (logo) at ([xshift=1.2cm, yshift=-1.5cm]current page.north west)
    {\includegraphics[width= 0.4\textwidth]{Figures/PP-PUT-WORD.png}};

    \node[inner sep=0pt, anchor=center] (logo2) at ([xshift=-1.6cm, yshift=1.7cm]current page.south east)
    {\includegraphics[width= 2.2cm]{Figures/PP-PUT-WIIT-LOGO.png}};


    \draw [double distance=4mm,
           double=gray,
           draw opacity=0,
           rotate=150,
           anchor=center,
           postaction={
                decorate,
                decoration={
                      raise=-1ex,
                      text along path, 
                      reverse path,
                      text align={fit to path stretching spaces},
                      text={|\ttfamily\footnotesize\color{black}|Kierunek\space Informatyka\space |\ttfamily\footnotesize\color{gray}|Wydzial\space Informatyki\space i\space Telekomunikacji}
                }
           }
        ] (logo2.center) circle (1.4cm);
    
  \end{tikzpicture}
	\section*{WYKRESY ZBIORCZE }
	%Wykresy, obrazy itd, warto opakowywać w `figure`, dzięki czemu można dodać Caption\Opis do figury, jak i `label` dzięki któremu później można odwoływać się do figur np. ``
	\begin{figure}[H]
		\centering
		\label{fig:enter-label}
		%Tworzę "pojemnik" na wykres `pgfplots` tak żeby automatycznie wyskalował mi wygenerowany wykres na szerokość strony. 
		\noindent\resizebox{\textwidth}{!}{
			\begin{tikzpicture}
				\begin{axis}[%
					name=plotA, anchor=left of south west,
					title={Heap Sort}, 
					xlabel={Rozmiar instancji}, ylabel={Czas(s)}, 
					legend pos=north west,
					xmode = log, log basis x={2},
					every axis plot post/.style={very thick},
					/tikz/plot label/.style={black, anchor=west}
					]
					\addplot[red, dashed, smooth] table[x=InputSize,y=Time,meta=Algorithm,col sep=comma] {heap_sort_a_shaped_array.txt};
					\addplot[color=blue, dashed, smooth] table[x=InputSize,y=Time,meta=Algorithm,col sep=comma] {heap_sort_constant_array.txt};
					\addplot[green, dotted, smooth] table[x=InputSize,y=Time,meta=Algorithm,col sep=comma] {heap_sort_decreasing_array.txt};
					\addplot[black, dotted, smooth] table[x=InputSize,y=Time,meta=Algorithm,col sep=comma] {heap_sort_increasing_array.txt};
					\addplot[yellow, dashed, smooth] table[x=InputSize,y=Time,meta=Algorithm,col sep=comma] {heap_sort_random_array.txt};
					\legend{A Shaped,Constant,Decreasing,Increasing,Random}
				\end{axis}
				\begin{axis}[%
					title={Insertion Sort}, 
					name=plotB, at=(plotA.right of south east), 
					anchor=left of south west,
					xlabel={Rozmiar instancji}, ylabel={Czas(s)}, 
					legend pos=north west,
					xmode = log, log basis x={2}, %Ustawiam oś x na logarytmiczną (log2)
					every axis plot post/.style={very thick},
					/tikz/plot label/.style={black, anchor=west}
					]
					\addplot[red, dashed, smooth] table[x=InputSize,y=Time,meta=Algorithm,col sep=comma] {insertion_sort_a_shaped_array.txt};
					\addplot[color=blue, dashed, smooth] table[x=InputSize,y=Time,meta=Algorithm,col sep=comma] {insertion_sort_constant_array.txt};
					\addplot[green, dotted, smooth] table[x=InputSize,y=Time,meta=Algorithm,col sep=comma] {insertion_sort_decreasing_array.txt};
					\addplot[black, dotted, smooth] table[x=InputSize,y=Time,meta=Algorithm,col sep=comma] {insertion_sort_increasing_array.txt};
					\addplot[yellow, dashed, smooth] table[x=InputSize,y=Time,meta=Algorithm,col sep=comma] {insertion_sort_random_array.txt};
					\legend{A Shaped,Constant,Decreasing,Increasing,Random}
				\end{axis}
				\begin{axis}[%
					title={Quick Sort Left Pivot}, 
					name=plotD, at=(plotB.below south west), 
					anchor=above north west,
					xlabel={Rozmiar instancji}, ylabel={Czas(s)}, 
					legend pos=north west,
					xmode = log, log basis x={2},
					every axis plot post/.style={very thick},
					/tikz/plot label/.style={black, anchor=west}
					]
					\addplot[red, dashed, smooth] table[x=InputSize,y=Time,meta=Algorithm,col sep=comma] {quick_sort_left_pivot_a_shaped_array.txt};
					\addplot[color=blue, dashed, smooth] table[x=InputSize,y=Time,meta=Algorithm,col sep=comma] {quick_sort_left_pivot_constant_array.txt};
					\addplot[green, dotted, smooth] table[x=InputSize,y=Time,meta=Algorithm,col sep=comma] {quick_sort_left_pivot_decreasing_array.txt};
					\addplot[black, dotted, smooth] table[x=InputSize,y=Time,meta=Algorithm,col sep=comma] {quick_sort_left_pivot_increasing_array.txt};
					\addplot[yellow, dashed, smooth] table[x=InputSize,y=Time,meta=Algorithm,col sep=comma] {quick_sort_left_pivot_random_array.txt};
					\legend{A Shaped,Constant,Decreasing,Increasing,Random}
				\end{axis}
				\begin{axis}[%
					title={Quick Sort Random Pivot},
					name=plotC, at=(plotD.left of south west), 
					anchor=right of south east,
					xlabel={Rozmiar instancji}, ylabel={Czas(s)}, 
					legend pos=north west,
					xmode = log, log basis x={2},
					every axis plot post/.style={very thick},
					/tikz/plot label/.style={black, anchor=west}
					]
					\addplot[red, dashed, smooth] table[x=InputSize,y=Time,meta=Algorithm,col sep=comma] {quick_sort_random_pivot_a_shaped_array.txt};
					\addplot[color=blue, dashed, smooth] table[x=InputSize,y=Time,meta=Algorithm,col sep=comma] {quick_sort_random_pivot_constant_array.txt};
					\addplot[green, dotted, smooth] table[x=InputSize,y=Time,meta=Algorithm,col sep=comma] {quick_sort_random_pivot_decreasing_array.txt};
					\addplot[black, dotted, smooth] table[x=InputSize,y=Time,meta=Algorithm,col sep=comma] {quick_sort_random_pivot_increasing_array.txt};
					\addplot[yellow, dashed, smooth] table[x=InputSize,y=Time,meta=Algorithm,col sep=comma] {quick_sort_random_pivot_random_array.txt};
					\legend{A Shaped,Constant,Decreasing,Increasing,Random}
				\end{axis}
				\begin{axis}[
					title={Selection Sort},
					title style={yshift=5pt},
					name=plotE, at=(plotD.below south west),
					%yshift=-0.8cm,
					%xshift=2.1cm,
					anchor=above north west,
					%	yticklabel style={xshift=-1cm},
					xlabel={Rozmiar instancji}, ylabel={Czas(s)},
					xmode = log, log basis x={2},
					every axis plot post/.style={very thick},
					/tikz/plot label/.style={black, anchor=north}
					]
					\addplot[red, dashed, smooth] table[x=InputSize,y=Time,meta=Algorithm,col sep=comma] {selection_sort_a_shaped_array.txt};
					\addplot[color=blue, dashed, smooth] table[x=InputSize,y=Time,meta=Algorithm,col sep=comma] {selection_sort_constant_array.txt};
					\addplot[green, dotted, smooth] table[x=InputSize,y=Time,meta=Algorithm,col sep=comma] {selection_sort_decreasing_array.txt};
					\addplot[black, dotted, smooth] table[x=InputSize,y=Time,meta=Algorithm,col sep=comma] {selection_sort_increasing_array.txt};
					\addplot[yellow, dashed, smooth] table[x=InputSize,y=Time,meta=Algorithm,col sep=comma] {selection_sort_random_array.txt};
					\legend{A Shaped,Constant,Decreasing,Increasing,Random}
				\end{axis}
				\begin{axis}[
					title={Shell Sort Sedgewick},
					title style={yshift=5pt},
					name=plotF, at=(plotE.left of south west),
					%yshift=-0.8cm,
					%xshift=2.1cm,
					anchor=right of south east,
					%	yticklabel style={xshift=-1cm},
					xlabel={Rozmiar instancji}, ylabel={Czas(s)},
					xmode = log, log basis x={2},
					every axis plot post/.style={very thick},
					/tikz/plot label/.style={black, anchor=north}
					]
					\addplot[red, dashed, smooth] table[x=InputSize,y=Time,meta=Algorithm,col sep=comma] {shell_sort_sedgewick_a_shaped_array.txt};
					\addplot[color=blue, dashed, smooth] table[x=InputSize,y=Time,meta=Algorithm,col sep=comma] {shell_sort_sedgewick_constant_array.txt};
					\addplot[green, dotted, smooth] table[x=InputSize,y=Time,meta=Algorithm,col sep=comma] {shell_sort_sedgewick_decreasing_array.txt};
					\addplot[black, dotted, smooth] table[x=InputSize,y=Time,meta=Algorithm,col sep=comma] {shell_sort_sedgewick_increasing_array.txt};
					\addplot[yellow, dashed, smooth] table[x=InputSize,y=Time,meta=Algorithm,col sep=comma] {shell_sort_sedgewick_random_array.txt};
					\legend{A Shaped,Constant,Decreasing,Increasing,Random}
				\end{axis}
			\end{tikzpicture}
		}
	\end{figure}
\end{document